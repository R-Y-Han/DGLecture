\documentclass[12pt, a4paper]{ctexart}

\usepackage{amsmath}
\usepackage{array}
\usepackage{appendix}
\usepackage{listings}
\usepackage{xcolor}
\usepackage{fontspec}
\usepackage{underscore}
\setmonofont{Consolas}
\lstset{
	breaklines=true,
	language = C++, 
	numbers=left,
	backgroundcolor=\color{red!0!green!25!blue!25},%代码块背景色为浅灰色
	rulesepcolor= \color{gray}, %代码块边框颜色
	numberstyle= \small,%行号字体
	frame=shadowbox%用方框框住代码块
	frame=single
}

\title{间断有限元第一次作业报告}
\author{九所 $\quad$ 韩若愚}
\date{2022.3.4}
\begin{document}
	\maketitle
	
	\section{题目}
	Consider the following ODE problem:
	\begin{equation}
	\begin{cases}
	& u_x = 2 \pi \cos( 2 \pi x), \quad x \in [0,1]\\
	& u(0) = 0.
	\end{cases}
	\end{equation}
	
	The exact solution is $ u(x) = \sin(2 \pi x) $. Code up the first DG method for $k=1$ and $ k =2$ with uniform meshes. Show error tables for the numerical erro in $L_2$ norm.
	
	\section{算法}
	
	首先对单元$[0,1]$ 进行均匀剖分。假设将区间均匀剖分为$n$份,令:
	\begin{equation}
	0 = x_{\frac{1}{2}} < x_{\frac{3}{2}} < \dots < x_{n-\frac{1}{2}} < x_{n+\frac{1}{2}} = 1
	\end{equation}
	则第$j$个区间为:$ I_j = [x_{j- \frac{1}{2}}, x_{j + \frac{1}{2}}]$,每个区间的长度都为$h = \frac{1}{n}$。记$x_{j+1/2}^- = \lim_{x \in I_j, \, x \to x_{j+1/2}} \, x, \  x_{j+1/2}^+ = \lim_{x \in I_{j+1}, \, x \to x_{j+1/2}} \, x$。
	
	假设所求数值解$u_h$存在的空间为:$ V_h^k := \{ v : \  v|_{I_j} \in P^k(I_j), j = 1, \dots, n \}$,其中$k$为给定常数,$P^k(I_j)$为定义在$I_j$上的最高次项不超过$k$次的多项式空间,并假设检验函数$v \in V_h^k$,用$v$乘以方程两端并在$I_j$上积分,并保证$k=0$时算法退化为一阶有限差分格式,于是得到:
	\begin{equation}
	\begin{split}
	& - \int_{I_j} u_h \, v' \, dx + u_h(x_{j+1/2}^-) v(x_{j+1/2}^-) \\
	& \qquad= u_h(x_{j-1/2}^-) v(x_{j-1/2}^+) + \int_{I_j} f \, v \, dx, \quad j = 1, \dots, n
	\end{split}
	\label{1}
	\end{equation}
	
	在$I_j$上取定一组$P^k(I_j)$的基底$\{ \phi_j^l \}_{l=0}^k$,则数值解$u_h$在$I_j$上为:$u_h|_{I_j}(x) = \sum_{j=1}^n \sum_{l=0}^k \, u_j^l \, \phi_j^l(x)$,求解$u_h$即求解系数$u_j^l, \, j=1, \dots, n, \, l=0, \dots, k$。令检验函数$v = \phi_j^m, \, m=0, \dots, k$,则方程(\ref{1})变为:
	\begin{equation}
	\begin{split}
	&\sum_{l=0}^k \, u_j^l \big[ -\int_{I_j} \phi_j^l (\phi_j^m)' \, dx + \phi_j^l(x_{j+1/2}) \phi_j^m (x_{j+1/2}) \big]\\
	& \qquad  = u_h(x_{j-1/2}^-) \phi_j^m (x_{j-1/2}) + \int_{I_j} f \phi_j^m \, dx, \quad  j = 1, \dots, n
	\end{split}
	\label{2}
	\end{equation}
	这是一个在$I_j$上的线性方程组。
	
	为便于求解(\ref{2}),假设对固定的$l$,每个$I_j$上的基底$\phi_j^l, \, j=0, \dots, n$的自由度的定义形式都相同。现在假设参考单元$I=[0,1]$,$\Phi_j : I_j \to I$是从$I_j$到$I$的仿射变换,$\xi := \Phi_j(x) = (x- x_{j-1/2})/h, \, x \in I_j$。显然$\Phi_j$是$I_j$到$I$的微分同胚,$\Phi_j^{-1} (\xi) = \xi \cdot h + x_{j-1/2}$,$\Phi_j(I_j) = I$。令$\phi^l ( \xi ) = \phi_j^l \circ \Phi_j^{-1} (\xi) = \phi_j^l(x(\xi))$,因为每个区间$I_j$上$\phi_j^l$的自由度定义都相同,可以发现$\phi^l$的取法是和$j$无关的,同时$\{ \phi^l \}_{l=0}^k$也是$P^k(I)$的一组基底。考虑方程(\ref{2})中各项:
	\begin{equation}
	\begin{split}
	& \frac{d}{dx} (\phi_j^m (x)) = \frac{d}{d \xi} (\phi^m(\xi)) \cdot \frac{d \xi}{dx} = (\phi^m)' (\xi) \cdot \frac{d\xi}{dx}\\
	& \int_{I_j} \phi_j^l(x) (\phi_j^m)' (x) \, dx = \int_{I_j} \phi_j^l(x) \frac{d}{dx}(\phi_j^m (x)) \, dx\\
	& \quad = \int_{I} \phi^l (\xi) (\phi^m)' (\xi) \frac{d\xi}{dx} \, d\xi \cdot \frac{dx}{d\xi}\\
	& \quad = \int_{I} \phi^l(\xi) (\phi^m)'(\xi) \, d\xi
	\end{split}
	\end{equation}
	
	\begin{equation}
	\phi_j^l(x_{j+1/2}) \phi_j^m(x_{j+1/2}) = \phi^l(1) \phi^l (1)
	\end{equation}
	
	\begin{equation}
	\begin{split}
	& u_h(x_{j-1/2}^-) \phi_j^m(x_{j-1/2}) = u_h(x_{j-1/2}^-) \phi^m(0)\\
	& u_h(x_{j-1/2}^-) = 
	\begin{cases}
	& u(0), \quad j=1\\
	& \sum_l u_{j-1}^l \phi_{j-1}^l (x_{j-1/2}^-) = \sum_l u_{j-1}^l \phi^l(1), \quad j\ge 1
	\end{cases}
	\end{split}
	\label{3}
	\end{equation}
	
	\begin{equation}
	\begin{split}
	\int_{I_j} f(x) \phi_j^m (x) \, dx & = \int_I f(x(\xi)) \phi^m (\xi) \, d\xi \cdot \frac{dx}{d\xi} = h \cdot \int_I f(x(\xi)) \phi^m (\xi) \, d\xi\\
	x(\xi) & = \xi \cdot h +x_{j-1/2}
	\end{split}
	\end{equation}
	
	于是所有$I_j$上的运算可以转换为$I$上的运算,方程组(\ref{2})变为:
	\begin{equation}
	\begin{split}
	& \sum_{l=0}^n u_j^l \big[ - \int_{I} \phi^l (\phi^m)' \, d\xi + \phi^l(1) \phi^m(1) \big]\\
	& = u_h(x_{j-1/2}^-) \phi^m(0) + h \int_I f(\xi h + x_{j - 1/2})\phi^m(\xi) \, d\xi,\\
	& \quad j= 1, \dots , k
	\end{split}
	\label{4}
	\end{equation}
	其中$u_h(x_{j-1/2}^-)$由(\ref{3})给出。方程组(\ref{4})可写为矩阵形式:$A \textbf{u}_j = B_j$,其中$A$为$(k+1) \times (k+1)$维方阵,其第$(m,l)$个元素为$a_{ml} = - \int_{I} \phi^l (\phi^m)' \, d\xi + \phi^l(1) \phi^m(1)$,$\textbf{u}_j = (u_j^0, \dots, u_j^n)^T$为需要求解的列向量,$B_j$为$k+1$维列向量,第$m$个元素为$B_j^m = u_h(x_{j-1/2}^-) \phi^m(0) + h \int_I f(\xi h + x_{j - 1/2})\phi^m(\xi) \, d\xi$。
		
	最后在每个$I_j$上令:$\phi_j^0(x) = 1, \  \phi_j^1(x) = (x- x_{j-1/2})/(x_{j+1/2}-x_{j-1/2})$, $\phi_j^2(x) = (x- x_{j-1/2})^2/(x_{j+1/2}-x_{j-1/2})^2, \  x \in I_j$,则参考单元$I$上相应的基底为:$ \phi^0(\xi) = 1, \  \phi^1(\xi) = \xi, \ \phi^2(\xi) = \xi^2, \ \xi \in I$。通过简单计算能直接得到$A$的值如下:
	\begin{equation}
	\mbox{k=1时}: \quad A=
	\begin{bmatrix}
	1 & 1 \\
	0 & 1/2
	\end{bmatrix}
	\quad A^{-1} = 
	\begin{bmatrix}
	1 & -2 \\
	0 & 2
	\end{bmatrix}
	\end{equation}
	
	\begin{equation}
	\mbox{k=2时}: \quad A = 
	\begin{bmatrix}
	1 & 1 & 1 \\
	0 & 1/2 & 2/3 \\
	0 & 1/3 & 1/2
	\end{bmatrix}
	\quad A^{-1} = 
	\begin{bmatrix}
	1 & -6 & 6\\
	0 & 18 & -24\\
	0 & -12 & 18
	\end{bmatrix}
	\end{equation}
	对于$B_j$中的积分项用复化梯形公式计算。
	
	\section{计算结果}
	
	将$I_j$上的基底$\phi_j^l$延拓到整个区间:令
	$$
	\varphi_j^l (x) = 
	\begin{cases}
	& \phi_j^l(x), \quad x \in I_j\\
	& 0, \quad else
	\end{cases}
	$$
	则数值解$u_h(x) = \sum_{j=0}^n \sum_{l=0}^k \, u_j^l \varphi_j^l(x)$。假设真解为$u$,则$L^2$误差为:
	\begin{equation}
	\begin{split}
	e & = (\int_0^1 |u - u_h|^2 \, dx)^{1/2} = (\sum_{j=0}^n \int_{I_j} |u(x)-\sum_{j=0}^k u_j^l \varphi_j^l(x) |^2 \, dx )^{1/2}\\
	& = (\sum_{j=0}^n \int_{I_j} |u(x) - \sum_{l=0}^k u_j^l \phi_j^l(x)|^2 \, dx )^{1/2}\\
	& = (\sum_{j=0}^n \int_{I_j} |u(x) - \sum_{l=0}^k u_j^l \phi^l (\xi(x)) |^2 \, dx )^{1/2}
	\end{split}
	\end{equation}
	
	在计算积分时使用梯形公式数值计算,将每个$I_j$上的积分$\int_{I_j} |u(x) - \sum_{l=0}^k u_j^l \phi^l (\xi(x)) |^2 \, dx$近似为
	$$
	( |u(x_{j+1/2}) - \sum_l u_j^l \phi^l(1) |^2 + | u(x_{j-1/2}) - \sum_l u_j^l \phi^l(0) |^2 ) \cdot h /2
	$$
	于是得到误差表如下(保留三位小数):
	\begin{table}[htbp]
		\centering
		\caption{$k=1$}
		\begin{tabular}{| p{80pt}<{\centering} | p{80pt}<{\centering} | p{80pt}<{\centering} |}
			\hline
			n & $L^2$ error & order\\
			\hline
			10 & 3.306e-2 &  \\
			\hline
			20 & 8.327e-3 & 1.989 \\
			\hline
			40 & 2.086e-3 & 1.997 \\
			\hline
			80 & 5.217e-4 & 1.999 \\
			\hline
			160 & 1.304e-4 & 2.000 \\
			\hline
			320 & 3.261e-5 & 2.000 \\
			\hline
		\end{tabular}
	\end{table}

	\begin{table}[htbp]
		\centering
		\caption{$k=2$}
		\begin{tabular}{| p{80pt}<{\centering} | p{80pt}<{\centering} | p{80pt}<{\centering} |}
			\hline
			n & $L^2$ error & order\\
			\hline
			10 & 2.051e-3 &  \\
			\hline
			20 & 2.573e-4 & 2.995 \\
			\hline
			40 & 3.197e-5 & 3.009 \\
			\hline
			80 & 3.880e-6 & 3.043 \\
			\hline
			160 & 4.261e-7 & 3.187 \\
			\hline
			320 & 2.381e-8 & 4.162 \\
			\hline
		\end{tabular}
	\end{table}

	\section{分析}
	
	通过误差表可以看到当$k=1$时收敛阶稳定在二阶,$k=2$时收敛阶在三阶附近。而且步长相同时,$k=2$得到的误差比$k=1$得到的误差小了几个数量级,$k=2$时得到的结果更为精确。
	
	对真解$u(x)= \sin(2\pi x)$在$x = x_{j-1/2}$处Taylor展开:
	\begin{equation}
	\begin{split}
	u(x) = & \sin(2\pi x_{j-1/2}) + 2\pi \cos(2\pi x_{j-1/2}) (x - x_{j-1/2})\\
	& - 2\pi^2 \sin(2\pi x_{j-1/2}) (x- x_{j-1/2})^2 + O((x- x_{j-1/2})^3)
	\end{split}
	\label{taylor}
	\end{equation}
	注意到$I_j$上基底$\{\phi_j^l\}_{l=0}^2$的定义形式,(\ref{taylor})变为:
	\begin{equation}
	\begin{split}
	u(x) = & \sin(2\pi x_{j-1/2}) \phi_j^0(x) + 2\pi \cos(2\pi x_{j-1/2}) h \phi_j^1(x)\\
	& - 2\pi^2 \sin(2\pi x_{j-1/2}) h^2 \phi_j^2(x) + O((x- x_{j-1/2})^3)
	\end{split}
	\end{equation}
	
	$u$显然满足方程(\ref{1}),由于$v$的任意性,忽略$u$的无穷小量,应有:
	\begin{equation*}
	\begin{split}
	u_h(x) = & \sin(2\pi x_{j-1/2}) \phi_j^0(x) + 2\pi \cos(2\pi x_{j-1/2}) h \phi_j^1(x)\\
	& - 2\pi^2 \sin(2\pi x_{j-1/2}) h^2 \phi_j^2(x) \quad a.e.
	\end{split}
	\end{equation*}
	
	所以$k=1, \  k=2$时收敛阶分别应该是$2, 3$。计算结果是符合理论的。
	
	但是在数值求解$k=2$情况的时候可以看到阶数在不断变大,最后变成了4。实际上在计算$k=2$的情况时我发现,方程组$A\textbf{u}_j =B_j$中右端项$B_j$的一个小扰动在步长比较小($h <= 0.0125$)时对误差阶数会产生比较大的影响,我认为这是机器误差导致的,因为实际上当右端项扰动在1e-4左右时,对$L^2$范数误差的影响只在1e-7左右。虽然从结果上看对真解的逼近仍然非常好,但是误差的收敛阶却受到了比较大的影响。但是在$k=1$的情况下却不会出现这样的问题。
	
	
\end{document}